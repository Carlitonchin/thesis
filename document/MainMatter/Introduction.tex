\chapter*{Introducción}\label{chapter:introduction}
\addcontentsline{toc}{chapter}{Introducción}

Para cualquier institución es de vital importancia una buena comunicación con sus clientes, en muchas ocasiones las personas tienen dudas, o quieren presentar una queja, y aquellas empresas que logran manejar la situación y dejar a la persona satisfecha se separan del resto. En un mundo cada vez más globalizado que no para de innovar en materia de tecnología, esta comunicación ha sobrepasado la barrera de la distancia y son cada vez más las compañías que brindan una atención premium a sus clientes a través de internet. Pero esto no es una tarea fácil, se necesita en muchas ocasiones de personal capacitado y de una plataforma online capaz de organizar todo el proceso y evitar que se vuelva un caos.
\newline

\textbf{Problemática}
\newline

Todo usuario de la Universidad de La Habana en caso de presentar algún problema determinado debe contactar con su administrador de área personalmente, realizar una llamada telefónica o escribir a un buzón de atención a usuarios. La situación pandémica aún vigente nos ha demostrado la necesidad de ser capaces de atender situaciones fuera del recinto laboral e incluso fuera del horario laboral, mientras no existe un mecanismo automático que permita una interacción segura y estable entre el usuario y quien lo atenderá. Por otra parte, cada resolución puede implicar un determinado conjunto de informaciones laborales no registradas actualmente, como pueden ser los elementos que formaron parte de la solución.\newline

\textbf{Motivación}
\newline

El tiempo ha demostrado que el uso de aplicaciones de comunicación como Telegram o Whatsapp no es suficiente, en muchas ocasiones las dudas se pierden entre cientos de mensajes en los grupos, a veces la problemática no llega a la persona adecuada, o se sobrecargan otros con un montón de preguntas con las que no pueden lidiar en un tiempo razonable para el estudiante. Por ello surge la necesidad de crear una plataforma digital que pueda resolver este problema, una en la que los estudiantes puedan escribir sus dudas, y mediante un mecanismo eficiente estos puedan obtener una respuesta lo más pronto posible.
\newline

\textbf{Objetivos}
\newline

Se debe crear una plataforma que cumpla con los siguientes requisitos:
\begin{itemize}
	\item Debe ser de fácil acceso para los estudiantes.
	
	\item Los estudiantes deben poder escribir sus dudas.
	
	\item Las dudas deben ser respondidas rápidamente.
	
	\item Debe existir un mecanismo mediante el cuál el estudiante pueda dar más detalles sobre su duda en caso de ser necesario.
	
	\item Los estudiantes deben poder acceder al sistema sin importar el lugar en que se encuentren.
\end{itemize}