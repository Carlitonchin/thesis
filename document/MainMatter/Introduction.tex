\chapter*{Introducción}\label{chapter:introduction}
\addcontentsline{toc}{chapter}{Introducción}

Todo usuario de la Universidad en caso de presentar algún problema determinado debe contactar con su administrador de área personalmente, realizar una llamada telefónica o escribir a un buzón de atención a usuarios. La situación pandémica aún vigente nos ha demostrado la necesidad de ser capaces de atender situaciones fuera del recinto laboral e incluso fuera del horario laboral, mientras no existe un mecanismo automático que permita una interacción segura y estable entre el usuario y quien lo atenderá. Por otra parte, cada resolución puede implicar un determinado conjunto de informaciones laborales no registradas actualmente, como pueden ser los elementos que formaron parte de la solución.\newline

Por ello surge la necesidad de crear una plataforma digital que pueda resolver este problema, una plataforma en que los estudiantes puedan escribir sus dudas, y mediante un mecanismo eficiente estos puedan obtener una respuesta lo más pronto posible.
\newline

\textbf{Objetivos del sistema:}

\begin{itemize}
	\item Debe ser de fácil acceso para los estudiantes.
	
	\item Los estudiantes deben poder escribir sus dudas.
	
	\item Las dudas deben ser respondidas rápidamente.
	
	\item Debe existir un mecanismo mediante el cuál el estudiante pueda dar más detalles sobre su duda en caso de ser necesario.
	
	\item Los estudiantes deben poder acceder al sistema sin importar en el lugar que se encuentren.
\end{itemize}