\chapter{Estado del Arte}\label{chapter:state-of-the-art}

A continuación se presentan un conjunto de artículos y/o herramientas punteras a día de hoy en términos de sistemas de resolución de dudas, además de los principales lenguajes, bibliotecas y frameworks utilizados en el mundo del desarrollo de software.

\section{Sistemas de Recuperación de información}

La \textbf{recuperación de información} en computación es el proceso de obtener recursos relevantes dentro de una colección de documentos dada una consulta \cite{ir}.
\newline

Los sistemas de recuperación de información pueden ser de texto, imágenes, videos, audios, etc. A día son utilizados por millones de usuarios cada día. El motor de búsqueda de \href{google.com}{Google}, el buscador de \href{youtube.com}{Youtube}, son ejemplo de esto.
\newline

Estos sistemas también son usados en el desarrollo de aplicaciones automatizadas de respuesta a preguntas. (veremos algunos ejemplos de esto más adelante)

\section{Hugging Face}

\textbf{Hugging Face Inc.} es una compañía norteamericana especializada en el desarrollo de herramientas y aplicaciones con el uso de Machine Learning \cite{hugging_face}.
\newline

Hugging Face Hub es una plataforma donde los usuarios pueden compartir datasets, modelos de machine learning, para que puedan ser usados y/o mejorados por la comunidad \cite{hugging_face_hub}.

\section{Procesamiento de lenguaje natural}

El \textbf{procesamiento de lenguaje natural} es un campo de la inteligencia artificial que estudia la interacción entre las computadoras y el lenguaje humano. Se encarga de la investigación y desarrollo de mecanismos para la comunicación entre humanos y máquinas por medio del lenguaje natural \cite{nlp}.
\newline

\textbf{BERT:} Bidirectional Encoder Representations from Transformers (BERT) es un modelo del lenguaje de machine learning desarrollado por \href{google.com}{Google} y publicado en 2018 \cite{bert}. El árticulo que describió la investigación de BERT ganó el Best Long Paper Award en la Conferencia Anual del 2019 de la North American Chapter of the Association for Computational Linguistics (NAACL) \cite{bert_award}.
\newline

\textbf{GPT-3:} Generative Pre-trained Transformer 3 (GPT-3) es un modelo del lenguaje desarrollado por \href{https://openai.com/}{Open AI} capaz de generar textos que simulan la redacción humana \cite{gpt}. Este poderoso modelo de 175 000 millones de parámetro desde su creación se ha destacado por generar textos de alta calidad.

\section{NSIR}

\textbf{NSIR} es un sistema desarrollado por la Universidad de Michigan que responde preguntas de forma automática. Dada una pregunta \textbf{NSIR} obtiene los principales resultados devueltos por motores de búsqueda (\href{google.com}{Google}, \href{yahoo.com}{Yahoo}, etc), los analiza y extrae de estos un conjunto de posibles respuestas, luego estas son evaluadas usando novedosas métricas, y de ahí retornan al usuario las respuestas con mejor calificación \cite{nsir}.

\section{Schema2QA}

\textbf{Schema2QA} es una herramienta de código abierto para generar sistemas de preguntas y respuestas a partir de un esquema de base de datos con unas pocas anotaciones sobre sus campos \cite{s2qa}. Este sistema intenta cubrir el espacio de consultas con un gran número de preguntas del dominio. Luego los datos obtenidos son usados para entrenar un modelo del lenguaje basado en \textit{BERT} \cite{bert}. Este sistema ha sido utilizado en dominios como los restaurantes, libros, música, entre otros, y alcanzado muy buenos resultados \cite{s2qa}.


\section{SmartQ}

\textbf{SmartQ} es un sistema de preguntas y respuestas basado en la reputación de los usuarios. Propone un mecanismo inteligente en la que hay usuarios especializados en diversas temáticas, y cuando alguien hace una pregunta esta es clasificada y enviada a personas expertas en el tema. Los usuarios van construyendo una reputación a medida que aportan respuestas valiosas a lo largo del tiempo. En el sistema también existe un mecanismo de seguidores en el que algunos usuarios expresan su confianza en otros, además de un sistema anti spam \cite{smartq}.

\section{Desarrollo de software} 

El mundo en que vivimos se caracteriza por estar cada vez más conectado, por lo que las aplicaciones web no paran de crecer, estas a día de hoy son la opción número 1 a la hora de desarrollar software que debe ser consumido por un gran número de usuarios.

\section{API REST}
\textbf{API:} Las API (interfaz de programación de aplicaciones) son mecanismos que permiten a dos componentes de software comunicarse entre sí mediante un conjunto de definiciones y protocolos \cite{apirest}.
\newline

\textbf{API REST:} REST significa transferencia de estado representacional. REST define un conjunto de funciones como GET, PUT, DELETE, etc. que los clientes pueden utilizar para acceder a los datos del servidor. Los clientes y los servidores intercambian datos mediante HTTP \cite{apirest}.
\newline

Las API REST son las más utilizadas en el desarrollo web a día de hoy.

\section{Frontend}
Los días en que bastaba \textit{HTML}, \textit{CSS} y \textit{Javascript Vanilla} (Javascript sin bibliotecas ni frameworks de terceros) para construir el frontend de una aplicación quedaron atrás. La complejidad de los productos actuales ha hecho que sea prácticamente imposible mantener software con estas tecnologías, por lo que los frameworks y bibliotecas de Javascript son prácticamente obligatorios a la hora de desarrollar una buena aplicación.
\newline

Según una encuesta desarrollada por StackOverflow en 2022 los frameworks y/o bibliotecas más usados en el desarrollo frontend son: \textit{React.js}, \textit{jQuery}, \textit{Angular} y \textit{Vue.js} \cite{encuesta2022}.

\subsection{Vue.js}
El framework utilizado para desarrollar el sistema fue \textit{Nuxt.js} (framework de \textit{Vue.js}) por todas las ventajas que proporciona.
\newline

\textbf{Vue.js:} es un framework de Javascript para construir interfaces de usuario. Está construido sobre \textit{HTML}, \textit{CSS} y \textit{Javascript}, y provee un modelo de programación declarativo y basado en componentes que te ayuda a implementar soluciones de una manera eficiente \cite{vue}.
\newline

\textbf{Nuxt.js:} es un framework construido encima de \textit{Vue.js} que brinda ciertas ventajas tales como enrutamiento automático, renderizado híbrido (renderizado del lado del cliente y del lado del servidor), optimización para motores de búsqueda, entre otras \cite{nuxt}.

\subsection{Tailwind CSS}
\textit{Tailwind CSS} es un framework de \textit{CSS} orientado a utilidades que tiene varias clases \textit{CSS} predefinidas que ayudan a estilizar una página sin abandonar el \textit{HTML} \cite{tailwind}.

\section{Backend}
Según la encuesta de StackOverflow del 2022 \cite{encuesta2022} estos son los lenguajes de programación más usados: 
\begin{enumerate}
	\item JavaScript
	\item Python
	\item TypeScript
	\item Java
	\item Bash/Shell
	\item C\#
	\item C++
	\item PHP
	\item PowerShell
	\item Go
	\item Rust
\end{enumerate}
No se incluyeron en la lista lenguajes de marcado y estilos (\textit{HTML} y \textit{CSS}) ni lenguajes de consulta (\textit{SQL})
\newline

Para el backend de este proyecto nos decantamos por \textit{Go} por su eficiencia y facilidad a la hora de desarrollar concurrencia
\subsection{Go}
\textbf{Go:} es un lenguaje de programación de código abierto desarrollado por Google. Es expresivo, conciso, limpio, y eficiente. Su mecanismo de concurrencia hace fácil escribir programas con estas características. Compila rápido a código de máquina, tiene recolector de basura. Es rápido y estáticamente tipado \cite{golang}.
\newline

\textit{Go} es muy utilizado para desarrollar backend de aplicaciones web, tiene varios frameworks de este estilo tales como \href{https://gobuffalo.io/en/}{Buffalo}, \href{https://echo.labstack.com/}{Echo}, \href{https://www.flamingo.me/}{Flamingo}, \href{http://www.gorillatoolkit.org/}{Gorilla}, y \href{https://gin-gonic.com/}{Gin}. Para este proyecto nos decantamos por \textit{Gin}.
\newline

\textbf{Gin:} es un framework web desarrollado sobre \textit{Go}. Es muy rápido y provee muchas facilidades a la hora de desarrollar una \textit{REST API} \cite{gin}.

\section{Base de datos}
Según la encuesta de StackOverflow de 2022 \cite{encuesta2022} los motores de base de datos más utilizados son:
\begin{enumerate}
	\item MySQL
	\item PostgreSQL
	\item SQLite
	\item MongoDB
	\item Microsoft SQL Server
	\item Redis
\end{enumerate}

Para este proyecto estaremos utilizando \textit{PostgreSQL} para almacenar los datos del sistema y \textit{Redis} para almacenar tokens de usuarios.
\newline

\textbf{PostgreSQL:} es un sistema de base de datos relacional muy poderoso y de código abierto con más de 35 años de desarrollo activo \cite{postgres}.
\newline

\textbf{Redis:} es una base de datos en memoria de código abierto usada principalmente como caché, motor de streaming, bróker de mensajes, entre otros usos \cite{redis}.

\subsection{ORM}
Un \textit{ORM} (Object–relational mapping) es una técnica de programación para convertir datos entre sistemas de tipos usando lenguajes de programación orientado a objetos. Esto crea una base de datos virtual que puede ser usada desde un lenguaje de programación \cite{orm}.
\newline

\textbf{Gorm:} es una biblioteca de ORM de \href{https://go.dev/}{Go} \cite{gorm}.

\section{JWT}

\textit{JSON Web Token} (\textit{JWT}) es un estándar abierto basado en JSON para la creación de tokens de acceso que permiten la propagación de identidad y privilegios \cite{jwt}.

\section{Docker}

\textit{Docker} es una herramienta que hace el desarrollo de una aplicación sea más eficiente y predecible, eliminando tareas de configuración repetitivas, para ello hace uso de \textit{contenedores} e \textit{imágenes} \cite{docker}.
\newline

\textbf{Contenedor:} en \textit{Docker} un contenedor no es más que un proceso que corre en una PC de forma aislada al resto de procesos \cite{docker_docs}.
\newline

\textbf{Imagen:} Una imagen de Docker es una plantilla de solo lectura que define su contenedor. La imagen contiene el código que se ejecutará, incluida cualquier definición para cualquier biblioteca o dependencia que el código necesite \cite{docker_docs}.
\newline

Durante el desarrollo del proyecto se utilizó \textit{Docker} por todas las ventajas que provee. Esta herramienta es la más amada por los desarrolladores según la encuesta de 2022 de \href{https://stackoverflow.com/}{StackOverflow} \cite{encuesta2022}.
