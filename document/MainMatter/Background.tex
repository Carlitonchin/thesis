\chapter{Estado del Arte}\label{chapter:state-of-the-art}

\section{Inteligencia Artificial}

La resolución de dudas es algo cotidiano, la mayoría de las empresas e instituciones lidian con las dudas de sus usuarios a diario, ya sea por redes sociales o por páginas oficiales. En muchas ocasiones, el volumen de preguntas es de tal magnitud que se necesitan a muchas personas pendientes, por lo que se han hecho estudios en pro de conseguir sistemas de resolución de dudas automatizados, los cuales hacen uso de una fusión entre sistemas de recuperación de información \cite{ir}, y modelos de procesamiento de lenguaje natural \cite{nlp}.
\newline

La \textbf{recuperación de información} \cite{ir} en computación es el proceso de obtener recursos relevantes dentro de una colección de documentos dada una consulta. Existen sistemas de este tipo que son usados cada día por millones de usuarios, para recuperar texto, imágenes, videos, audios, etc. El motor de búsqueda de \href{google.com}{Google} y el buscador de \href{youtube.com}{Youtube} son solo un par de ejemplos.
\newline

Estos sistemas ayudan, pero son insuficientes por sí solos, ya que sus usuarios deben revisar los documentos devueltos manualmente, verificar que en efecto era lo que estaban buscando, y luego encontrar o deducir en el contenido de los mismos la respuesta a su pregunta, lo que es una tarea tediosa sobre todo si el usuario no dispone del tiempo necesario para llevarla a cabo, y aún más teniendo en cuenta que la precisión de estos sistemas en la mayoría de las ocasiones deja mucho que desear.
\newline

Para solventar ese problema en los últimos años ha habido un auge en el \textbf{procesamiento de lenguaje natural} \cite{nlp}, el cual es un campo de la inteligencia artificial que estudia la interacción entre las computadores y el lenguaje humano, este se encarga del desarrollo de mecanismos para la comunicación entre humanos y máquinas por medio del lenguaje natural. Gracias a los esfuerzos realizados por diversas instituciones, se han creado varios modelos muy poderosos, entre los que destacan \textbf{BERT} \cite{bert} y \textbf{GPT-3} \cite{gpt}.
\newline

\textbf{BERT} (Bidirectional Encoder Representations from Transformers) es un modelo del lenguaje desarrollado por \href{google.com}{Google} y presentado en 2018 a través de un artículo, el cual fue merecedor del Best Long Paper
Award en la Conferencia Anual del 2019 de la North American Chapter of the Association for Computational Linguistics (NAACL) \cite{bert_award}.
\newline

\textbf{GPT-3} (Generative Pre-trained Transformer 3) es un modelo del lenguaje desarrollado por \href{https://openai.com/}{Open AI} capaz de generar textos que simulan la redacción humana. Este poderoso modelo de 175 000 millones de parámetro desde su creación se ha destacado por generar textos de alta calidad.
\newline

Entrenar estos modelos es una tarea titánica para la que se necesita una cantidad absurda de datos y hardware de última generación, por lo que se precisa de inversiones millonarias en infraestructura para poder conseguirlo.
\newline

En el año 2016 fue fundada \href{https://huggingface.co/}{Hugging Face} \cite{hugging_face}, una compañía especializada en el desarrollo de herramientas y aplicaciones con el uso de machine learning, y han lanzado una plataforma llamada \textbf{Hugging Face Hub} \cite{hugging_face_hub}, en la que son compartidos muchísimos datasets y modelos que son usados y mejorados por la comunidad. Esto ha democratizado de cierta manera el uso de los grandes avances del aprendizaje automático.
\newline

\section{Desarrollo de software}

Todos (o casi todos) los sistemas informáticos requieren de una interfaz amigable, una arquitectura, necesitan ser seguros, almacenar datos, y para ello se usan lenguajes de programación, bibliotecas, frameworks, motores de bases de datos y técnicas de desarrollo.
\newline

En la actualidad el desarrollo web está de moda por la facilidad que brinda de ser usado en cualquier dispositivo y en cualquier lugar, solo con un navegador y una conexión a internet. La mayoría de las aplicaciones más usadas a nivel mundial (\href{instagram.com}{Instagram}, \href{youtube.com}{Youtube}, \href{https://web.whatsapp.com/}{Whatsapp}) cuentan con una versión web.
\newline

Una aplicación web consta de 2 partes, el \textbf{Backend} y el \textbf{Frontend}. El \textbf{frontend} es la parte que corre en el dispositivo del usuario final, y se encarga de mostrar una interfaz con la que este puede interactuar para conseguir un objetivo. El \textbf{backend}, por otro lado, es la parte que corre en servidores y se encarga de la lógica del negocio, la seguridad, la persistencia de los datos, etc. El \textbf{frontend} interactúa con el \textbf{backend} para guardar, modificar, o acceder a información.
\newline

Los días en que bastaba \textit{HTML}, \textit{CSS} y \textit{Javascript Vanilla} (Javascript sin bibliotecas ni frameworks de terceros) para construir el frontend de una aplicación quedaron atrás. La complejidad de los productos actuales ha hecho que sea prácticamente imposible mantener software con estas tecnologías, por lo que los frameworks y bibliotecas de Javascript son prácticamente obligatorios a la hora de desarrollar una buena aplicación.
\newline

Según una encuesta desarrollada por StackOverflow \cite{encuesta2022} en 2022 los frameworks y/o bibliotecas más usados en el desarrollo frontend son: \textit{React.js}, \textit{jQuery}, \textit{Angular} y \textit{Vue.js}.
\newline

Para estilizar las páginas se utiliza el lenguaje \textit{CSS}, pero debido a la naturaleza del mismo, en aplicaciones grandes el código \textit{CSS} crece demasiado y se hace muy difícil de mantener, por eso en algunas ocasiones los desarrolladores optan por usar frameworks que eviten de cierta manera estas complicaciones, dos de los más usados en la actualidad son \href{getbootstrap.com}{Bootstrap} [cita], y \href{tailwindcss.com}{Tailwind CSS} \cite{tailwind}.
\newline

En el caso del backend existe una mayor variedad de tecnologías, por lo que los desarrolladores suelen elegir la que más se adecúa al proyecto que buscan desarrollar, en estos casos es casi imprescindible utilizar un framework, ya que las bases de código en aplicaciones no tan complejas suele crecer muchísimo por lo que puede haber problemas en el mantenimiento de las mismas y pueden surgir agujeros de seguridad.
\newline

Según la encuesta de StackOverflow del 2022 \cite{encuesta2022} estos son los lenguajes de programación más usados: 
\begin{enumerate}
	\item JavaScript
	\item Python
	\item TypeScript
	\item Java
	\item Bash/Shell
	\item C\#
	\item C++
	\item PHP
	\item PowerShell
	\item Go
	\item Rust
\end{enumerate}
No se incluyeron en la lista lenguajes de marcado y estilos (\textit{HTML} y \textit{CSS}) ni lenguajes de consulta (\textit{SQL})
\newline

Para almacenar los datos en una aplicación se utilizan motores de base de datos, según la encuesta de StackOverflow de 2022 \cite{encuesta2022} estos son los más utilizados:
\begin{enumerate}
	\item MySQL
	\item PostgreSQL
	\item SQLite
	\item MongoDB
	\item Microsoft SQL Server
	\item Redis
\end{enumerate}

Al igual que pasa con \textit{CSS}, el código de base de datos en aplicaciones grandes suele crecer muchísimo, hasta el punto que se convierte prácticamente en otro proyecto para mantener, para resolver este problema existen los ORM (Object–relational mapping) \cite{orm}, una técnica de programación para convertir datos entre sistemas de tipos usando lenguajes de programación orientado a objetos. Esto crea una base de datos virtual que puede ser usada desde un lenguaje de programación. Hay una gran cantidad de ORM's en todos (o casi todos) los lenguajes de programación.
\newline

Otro de los problemas habituales a la hora de desarrollar una aplicación son las diferencias entre los entornos de desarrollo y producción, en ocasiones las diferencias de sistemas operativos y/o versiones de paquetes instalados entre uno y el otro pueden causar inconvenientes. Afortunadamente hay una herramienta que hace que el desarrollo de una aplicación sea más eficiente y predecible, eliminando tareas de configuración repetitivas, estoy hablando de \href{docker.com}{Docker} \cite{docker_docs}. Según la encuesta de 2022 de Stackoverflow \cite{encuesta2022} Docker es la herramienta más amada por los desarrolladores.

\section{Sistemas de resolución de dudas relevantes}

Existen sistemas de resolución de preguntas que traen ideas novedosas que pueden ser tenidas en cuenta, a continuación se presentan algunos de ellos:

\subsection{NSIR}

\textbf{NSIR} es un sistema desarrollado por la Universidad de Michigan que responde preguntas de forma automática. Dada una pregunta \textbf{NSIR} obtiene los principales resultados devueltos por motores de búsqueda (\href{google.com}{Google}, \href{yahoo.com}{Yahoo}, etc), los analiza y extrae de estos un conjunto de posibles respuestas, luego estas son evaluadas usando novedosas métricas, y de ahí retornan al usuario las respuestas con mejor calificación \cite{nsir}.


\subsection{Schema2QA}

\textbf{Schema2QA} es una herramienta de código abierto para generar sistemas de preguntas y respuestas a partir de un esquema de base de datos con unas pocas anotaciones sobre sus campos \cite{s2qa}. Este sistema intenta cubrir el espacio de consultas con un gran número de preguntas del dominio. Luego los datos obtenidos son usados para entrenar un modelo del lenguaje basado en \textit{BERT} \cite{bert}. Este sistema ha sido utilizado en dominios como los restaurantes, libros, música, entre otros, y alcanzado muy buenos resultados \cite{s2qa}.


\subsection{SmartQ}

\textbf{SmartQ} es un sistema de preguntas y respuestas basado en la reputación de los usuarios. Propone un mecanismo inteligente en la que hay usuarios especializados en diversas temáticas, y cuando alguien hace una pregunta esta es clasificada y enviada a personas expertas en el tema. Los usuarios van construyendo una reputación a medida que aportan respuestas valiosas a lo largo del tiempo. En el sistema también existe un mecanismo de seguidores en el que algunos usuarios expresan su confianza en otros, además de un sistema anti spam \cite{smartq}.

