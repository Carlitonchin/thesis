\begin{resumen}
	La comunicación entre una institución y sus clientes es muy importante, se ve cada vez más frecuente el desarrollo de plataformas digitales que gestionan esta comunicación. Resolución de dudas, presentación de quejas y sugerencias son algunas de las tareas que estas plataformas ayudan a manejar.
	
	\mewline
	\
	
	En la Universidad de la Habana no existe a día de hoy una plataforma de estas características, y las dudas e inquietudes de los estudiantes se siguen atendiendo personalmente o por aplicaciones de chat como Telegram o Whatsapp, este mecanismo es ineficiente y los estudiantes y trabajadores del centro no están conformes.

	\newline
	\
	
	Existen varias herramientas de reportes e incidencias que son usados por las empresas para atender las inquietudes de sus clientes, pero la mayoría son bastante caras o no encajan con las necesidades de la universidad, por eso el presente trabajo propone la implementación de un sistema con algunas de las tecnologías punteras de desarrollo de software para solucionar el problema existente.
	
	\newline
	\
	
	La implementación propuesta es una solución web, con una API REST desarrollada en Gin, y el frontend en Nuxt. Es una plataforma en la que se registran los usuarios y los estudiantes de la universidad pueden escribir sus dudas, las cuales son clasificadas por algunos trabajadores y llegan a la persona que está capacitada para darle respuesta, resolviendo de esta manera la desorganización existente.
\end{resumen}

\begin{abstract}
	The communication between an institution and its clients is very important, currently it is more frequently the development of digital platforms that manage this communication. Resolution of doubts, presentation of complaints and suggestions are some
	of the tasks that these platforms help to handle.
	
	\newline
	\
	
	At the University of Havana there is no a platform with these characteristics, and the doubts and concerns of the students continue to be addressed
	personally or through chat applications such as Telegram or Whatsapp, this mechanism is inefficient and the students and workers of the institution are not satisfied.
	
	\newline
	\
	
	There are several reporting and incident systems that are used by companies to manage the concerns of their customers, but most are quite expensive
	or do not fit with the needs of the university, that is why the present work propses the implementation of a system with some of the leading technologies of
	software development to solve the existing problem.
	
	\newline
	\
	
	The proposed implementation is a web solution, with a REST API developed in Gin, and the frontend in Nuxt. It is a platform in which users register and university students can write their doubts, which are classified by some workers and reach the person who is trained to answer them, solving the existing disorganization.
\end{abstract}