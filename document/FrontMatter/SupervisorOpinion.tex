\begin{opinion}
    La gestión y administración de sistemas es uno de los pilares que sustentan la ya creciente explosión del uso de las tecnologías, la informatización y las comunicaciones. Si bien en sus variantes más modernas de software como servicio e infraestructura como servicio se libera de la gestión al usuario final, todavía se hace necesario el manejo de infraestructura y servicios críticos, ya sea sobre hardware propio o de terceros.
\newline
    
    En nuestro país, con la evolución del panorama tecnológico de los recientes años, se ha incrementado la demanda activa de personal cuyas habilidades y conocimientos puedan conducir el desarrollo tecnológico que la nación se ha propuesto alcanzar. Temas como la gestión de recursos humanos e inventario, la ciberseguridad y monitorización, la autenticación y autorización, gestión de dominio y la retroalimentación de cara a los usuarios finales se han convertido en las bases sobre las que se sustentan la sociedad informatizada que queremos construir. 
    \newline
    
    Dentro de estos objetivos, en particular, sobre la gestión los reportes e incidencias en la Universidad de La Habana se desarrolla la tesis del estudiante Carlos Alejandro Arrieta Montes de Oca. El estudiante propone una arquitectura de solución del problema de la atención a los planteamientos e inquietudes de los usuarios que hacen uso de la red institucional basada en 3 capas de atención por nivel de especialización. Dicha propuesta permite el manejo de forma distribuida de acuerdo al nivel de complejidad de la inquietud, dando una pronta respuesta de cara a los usuarios y aliviando la carga de trabajo de los especialistas principales.
    \newline
    
    Carlos Alejandro, en este documento de tesis, desarrolla una solución de software sobre Gin y Nuxt.js con una base de datos PostgreSQL. Dicha solución abarca todo el panorama de atención a usuarios y brinda una interfaz cómoda de comunicación entre cada usuario y el especialista u administrador correspondiente. Finalmente, realizó las pruebas de validación de los flujos de resolución de preguntas definidos y de carga de la base de datos alcanzando resultados satisfactorios.
    \newline
    
    El estudiante, durante el desarrollo de esta tesis, ha honrado de una forma u otra todos los conocimientos impartidos que se esperan de un graduado de nuestra institución. Ha realizado un estudio del estado del arte, consultando tanto documentación técnica como estudios científicos. Ha propuesto y ha llevado a cabo una arquitectura de solución, extensible, moderna y creativa al problema de la gestión de reportes e incidencias de la Universidad. Ha mostrado, también, las habilidades de comunicación necesarias para transmitir sus resultados. Finalmente, ha hecho gala de una fuerza de voluntad admirable que le ha permitido sobreponerse a todos los problemas presentados en la realización de esta tesis. 
    \newline
    
    Por estos motivos, considero que Carlos Alejandro Arrieta Montes de Oca ha demostrado con creces haber adquirido las habilidades que lo avalan como un excelente Científico de la Computación y estimo razonable solicitar al tribunal que se le otorgue la máxima calificación.
    \newline
    
    Solo me queda desearle el mayor de los éxitos en su vida profesional. Que siga honrando a nuestra institución como bien ha hecho en este documento y que siga cosechando los frutos de su esfuerzo donde sea que los vientos de la vida le terminen llevando.
\end{opinion}