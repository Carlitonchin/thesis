\documentclass[12pt,oneside]{uhthesis}
\usepackage{subfigure}
\usepackage[ruled,lined,linesnumbered,titlenumbered,algochapter,spanish,onelanguage]{algorithm2e}
\usepackage{amsmath}
\usepackage{cite}
\usepackage{amssymb}
\usepackage{amsbsy}
\usepackage{caption,booktabs}
\captionsetup{ justification = centering }
%\usepackage{mathpazo}
\usepackage{float}
\setlength{\marginparwidth}{2cm}
\usepackage{todonotes}
\usepackage{listings}
\usepackage{xcolor}
\usepackage{multicol}
\usepackage{graphicx}
\usepackage[spanish,es-noshorthands]{babel}

\floatstyle{plaintop}
\restylefloat{table}
\addbibresource{Bibliography.bib}
% \setlength{\parskip}{\baselineskip}%
\renewcommand{\tablename}{Tabla}
\renewcommand{\listalgorithmcfname}{Índice de Algoritmos}
%\dontprintsemicolon
\SetAlgoNoEnd

\definecolor{codegreen}{rgb}{0,0.6,0}
\definecolor{codegray}{rgb}{0.5,0.5,0.5}
\definecolor{codepurple}{rgb}{0.58,0,0.82}
\definecolor{backcolour}{rgb}{0.95,0.95,0.92}

\lstdefinestyle{mystyle}{
    backgroundcolor=\color{backcolour},   
    commentstyle=\color{codegreen},
    keywordstyle=\color{purple},
    numberstyle=\tiny\color{codegray},
    stringstyle=\color{codepurple},
    basicstyle=\ttfamily\footnotesize,
    breakatwhitespace=false,         
    breaklines=true,                 
    captionpos=b,                    
    keepspaces=true,                 
    numbers=left,                    
    numbersep=5pt,                  
    showspaces=false,                
    showstringspaces=false,
    showtabs=false,                  
    tabsize=4
}

\lstset{style=mystyle}

\title{Sistema de incidencias y reportes}
\author{\\\vspace{0.25cm}Carlos Alejandro Arrieta Montes de Oca}

\hypersetup{
	colorlinks=true,
	linkcolor=blue,
	filecolor=magenta,      
	urlcolor=cyan,
	pdftitle={Overleaf Example},
	pdfpagemode=FullScreen,
}

\advisor{\\\vspace{0.25cm}Lic. Roberto Marti Cedeño\\\vspace{0.2cm}Lic. Alexi Massó Muñoz}
\degree{Licenciado en  Ciencia de la Computación}
\faculty{Facultad de Matemática y Computación}
\date{Fecha: 28 de noviembre de 2022\\\vspace{0.25cm}\href{https://github.com/Carlitonchin/Tesis}{github.com/Carlitonchin/Tesis}}
\logo{Graphics/uhlogo}
\makenomenclature

\renewcommand{\vec}[1]{\boldsymbol{#1}}
\newcommand{\diff}[1]{\ensuremath{\mathrm{d}#1}}
\newcommand{\me}[1]{\mathrm{e}^{#1}}
\newcommand{\pf}{\mathfrak{p}}
\newcommand{\qf}{\mathfrak{q}}
%\newcommand{\kf}{\mathfrak{k}}
\newcommand{\kt}{\mathtt{k}}
\newcommand{\mf}{\mathfrak{m}}
\newcommand{\hf}{\mathfrak{h}}
\newcommand{\fac}{\mathrm{fac}}
\newcommand{\maxx}[1]{\max\left\{ #1 \right\} }
\newcommand{\minn}[1]{\min\left\{ #1 \right\} }
\newcommand{\lldpcf}{1.25}
\newcommand{\nnorm}[1]{\left\lvert #1 \right\rvert }

\begin{document}

\frontmatter
\maketitle

\begin{opinion}
    La gestión y administración de sistemas es uno de los pilares que sustentan la ya creciente explosión del uso de las tecnologías, la informatización y las comunicaciones. Si bien en sus variantes más modernas de software como servicio e infraestructura como servicio se libera de la gestión al usuario final, todavía se hace necesario el manejo de infraestructura y servicios críticos, ya sea sobre hardware propio o de terceros.
\newline
    
    En nuestro país, con la evolución del panorama tecnológico de los recientes años, se ha incrementado la demanda activa de personal cuyas habilidades y conocimientos puedan conducir el desarrollo tecnológico que la nación se ha propuesto alcanzar. Temas como la gestión de recursos humanos e inventario, la ciberseguridad y monitorización, la autenticación y autorización, gestión de dominio y la retroalimentación de cara a los usuarios finales se han convertido en las bases sobre las que se sustentan la sociedad informatizada que queremos construir. 
    \newline
    
    Dentro de estos objetivos, en particular, sobre la gestión los reportes e incidencias en la Universidad de La Habana se desarrolla la tesis del estudiante Carlos Alejandro Arrieta Montes de Oca. El estudiante propone una arquitectura de solución del problema de la atención a los planteamientos e inquietudes de los usuarios que hacen uso de la red institucional basada en 3 capas de atención por nivel de especialización. Dicha propuesta permite el manejo de forma distribuida de acuerdo al nivel de complejidad de la inquietud, dando una pronta respuesta de cara a los usuarios y aliviando la carga de trabajo de los especialistas principales.
    \newline
    
    Carlos Alejandro, en este documento de tesis, desarrolla una solución de software sobre Gin y Nuxt.js con una base de datos PostgreSQL. Dicha solución abarca todo el panorama de atención a usuarios y brinda una interfaz cómoda de comunicación entre cada usuario y el especialista u administrador correspondiente. Finalmente, realizó las pruebas de validación de los flujos de resolución de preguntas definidos y de carga de la base de datos alcanzando resultados satisfactorios.
    \newline
    
    El estudiante, durante el desarrollo de esta tesis, ha honrado de una forma u otra todos los conocimientos impartidos que se esperan de un graduado de nuestra institución. Ha realizado un estudio del estado del arte, consultando tanto documentación técnica como estudios científicos. Ha propuesto y ha llevado a cabo una arquitectura de solución, extensible, moderna y creativa al problema de la gestión de reportes e incidencias de la Universidad. Ha mostrado, también, las habilidades de comunicación necesarias para transmitir sus resultados. Finalmente, ha hecho gala de una fuerza de voluntad admirable que le ha permitido sobreponerse a todos los problemas presentados en la realización de esta tesis. 
    \newline
    
    Por estos motivos, considero que Carlos Alejandro Arrieta Montes de Oca ha demostrado con creces haber adquirido las habilidades que lo avalan como un excelente Científico de la Computación y estimo razonable solicitar al tribunal que se le otorgue la máxima calificación.
    \newline
    
    Solo me queda desearle el mayor de los éxitos en su vida profesional. Que siga honrando a nuestra institución como bien ha hecho en este documento y que siga cosechando los frutos de su esfuerzo donde sea que los vientos de la vida le terminen llevando.
\end{opinion}
\begin{resumen}
	La comunicación entre una institución y sus clientes es muy importante, se ve cada vez más frecuente el desarrollo de plataformas digitales que gestionan esta comunicación. Resolución de dudas, presentación de quejas y sugerencias son algunas de las tareas que estas plataformas ayudan a manejar.
	
	\mewline
	\
	
	En la Universidad de la Habana no existe a día de hoy una plataforma de estas características, y las dudas e inquietudes de los estudiantes se siguen atendiendo personalmente o por aplicaciones de chat como Telegram o Whatsapp, este mecanismo es ineficiente y los estudiantes y trabajadores del centro no están conformes.

	\newline
	\
	
	Existen varias herramientas de reportes e incidencias que son usados por las empresas para atender las inquietudes de sus clientes, pero la mayoría son bastante caras o no encajan con las necesidades de la universidad, por eso el presente trabajo propone la implementación de un sistema con algunas de las tecnologías punteras de desarrollo de software para solucionar el problema existente.
\end{resumen}

\begin{abstract}
	The communication between an institution and its clients is very important, currently it is more frequently the development of digital platforms that manage this communication. Resolution of doubts, presentation of complaints and suggestions are some
	of the tasks that these platforms help to handle.
	
	\newline
	\
	
	At the University of Havana there is no a platform with these characteristics, and the doubts and concerns of the students continue to be addressed
	personally or through chat applications such as Telegram or Whatsapp, this mechanism is inefficient and the students and workers of the institution are not satisfied.
	
	\newline
	\
	
	There are several reporting and incident systems that are used by companies to manage the concerns of their customers, but most are quite expensive
	or do not fit with the needs of the university, that is why the present work propses the implementation of a system with some of the leading technologies of
	software development to solve the existing problem.
\end{abstract}
\include{FrontMatter/Contents}

\mainmatter

\chapter*{Introducción}\label{chapter:introduction}
\addcontentsline{toc}{chapter}{Introducción}

Para cualquier institución es de vital importancia una buena comunicación con sus clientes, en muchas ocasiones las personas tienen dudas, o quieren presentar una queja, y aquellas empresas que logran manejar la situación y dejar a la persona satisfecha se separan del resto. En un mundo cada vez más globalizado que no para de innovar en materia de tecnología, esta comunicación ha sobrepasado la barrera de la distancia y son cada vez más las compañías que brindan una atención premium a sus clientes a través de internet. Pero esto no es una tarea fácil, se necesita en muchas ocasiones de personal capacitado y de una plataforma online capaz de organizar todo el proceso y evitar que se vuelva un caos.
\newline

\textbf{Problemática}
\newline

Todo usuario de la Universidad de La Habana en caso de presentar algún problema determinado debe contactar con su administrador de área personalmente, realizar una llamada telefónica o escribir a un buzón de atención a usuarios. La situación pandémica aún vigente nos ha demostrado la necesidad de ser capaces de atender situaciones fuera del recinto laboral e incluso fuera del horario laboral, mientras no existe un mecanismo automático que permita una interacción segura y estable entre el usuario y quien lo atenderá. Por otra parte, cada resolución puede implicar un determinado conjunto de informaciones laborales no registradas actualmente, como pueden ser los elementos que formaron parte de la solución.\newline

\textbf{Motivación}
\newline

El tiempo ha demostrado que el uso de aplicaciones de comunicación como Telegram o Whatsapp no es suficiente, en muchas ocasiones las dudas se pierden entre cientos de mensajes en los grupos, a veces la problemática no llega a la persona adecuada, o se sobrecargan otros con un montón de preguntas con las que no pueden lidiar en un tiempo razonable para el estudiante. Por ello surge la necesidad de crear una plataforma digital que pueda resolver este problema, una en la que los estudiantes puedan escribir sus dudas, y mediante un mecanismo eficiente estos puedan obtener una respuesta lo más pronto posible.
\newline

\textbf{Objetivos}
\newline

Se debe crear una plataforma que cumpla con los siguientes requisitos:
\begin{itemize}
	\item Debe ser de fácil acceso para los estudiantes.
	
	\item Los estudiantes deben poder escribir sus dudas.
	
	\item Las dudas deben ser respondidas rápidamente.
	
	\item Debe existir un mecanismo mediante el cuál el estudiante pueda dar más detalles sobre su duda en caso de ser necesario.
	
	\item Los estudiantes deben poder acceder al sistema sin importar el lugar en que se encuentren.
\end{itemize}
\chapter{Estado del Arte}\label{chapter:state-of-the-art}

\section{Inteligencia Artificial}

La resolución de dudas es algo cotidiano, la mayoría de las empresas e instituciones lidian con las dudas de sus usuarios a diario, ya sea por redes sociales o por páginas oficiales. En muchas ocasiones, el volumen de preguntas es de tal magnitud que se necesitan a muchas personas pendientes, por lo que se han hecho estudios en pro de conseguir sistemas de resolución de dudas automatizados, los cuales hacen uso de una fusión entre sistemas de recuperación de información \cite{ir}, y modelos de procesamiento de lenguaje natural \cite{nlp}.
\newline

La \textbf{recuperación de información} \cite{ir} en computación es el proceso de obtener recursos relevantes dentro de una colección de documentos dada una consulta. Existen sistemas de este tipo que son usados cada día por millones de usuarios, para recuperar texto, imágenes, videos, audios, etc. El motor de búsqueda de \href{google.com}{Google} y el buscador de \href{youtube.com}{Youtube} son solo un par de ejemplos.
\newline

Estos sistemas ayudan, pero son insuficientes por sí solos, ya que sus usuarios deben revisar los documentos devueltos manualmente, verificar que en efecto era lo que estaban buscando, y luego encontrar o deducir en el contenido de los mismos la respuesta a su pregunta, lo que es una tarea tediosa sobre todo si el usuario no dispone del tiempo necesario para llevarla a cabo, y aún más teniendo en cuenta que la precisión de estos sistemas en la mayoría de las ocasiones deja mucho que desear.
\newline

Para solventar ese problema en los últimos años ha habido un auge en el \textbf{procesamiento de lenguaje natural} \cite{nlp}, el cual es un campo de la inteligencia artificial que estudia la interacción entre las computadores y el lenguaje humano, este se encarga del desarrollo de mecanismos para la comunicación entre humanos y máquinas por medio del lenguaje natural. Gracias a los esfuerzos realizados por diversas instituciones, se han creado varios modelos muy poderosos, entre los que destacan \textbf{BERT} \cite{bert} y \textbf{GPT-3} \cite{gpt}.
\newline

\textbf{BERT} (Bidirectional Encoder Representations from Transformers) es un modelo del lenguaje desarrollado por \href{google.com}{Google} y presentado en 2018 a través de un artículo, el cual fue merecedor del Best Long Paper
Award en la Conferencia Anual del 2019 de la North American Chapter of the Association for Computational Linguistics (NAACL) \cite{bert_award}.
\newline

\textbf{GPT-3} (Generative Pre-trained Transformer 3) es un modelo del lenguaje desarrollado por \href{https://openai.com/}{Open AI} capaz de generar textos que simulan la redacción humana. Este poderoso modelo de 175 000 millones de parámetro desde su creación se ha destacado por generar textos de alta calidad.
\newline

Entrenar estos modelos es una tarea titánica para la que se necesita una cantidad absurda de datos y hardware de última generación, por lo que se precisa de inversiones millonarias en infraestructura para poder conseguirlo.
\newline

En el año 2016 fue fundada \href{https://huggingface.co/}{Hugging Face} \cite{hugging_face}, una compañía especializada en el desarrollo de herramientas y aplicaciones con el uso de machine learning, y han lanzado una plataforma llamada \textbf{Hugging Face Hub} \cite{hugging_face_hub}, en la que son compartidos muchísimos datasets y modelos que son usados y mejorados por la comunidad. Esto ha democratizado de cierta manera el uso de los grandes avances del aprendizaje automático.
\newline

\section{Desarrollo de software}

Todos (o casi todos) los sistemas informáticos requieren de una interfaz amigable, una arquitectura, necesitan ser seguros, almacenar datos, y para ello se usan lenguajes de programación, bibliotecas, frameworks, motores de bases de datos y técnicas de desarrollo.
\newline

En la actualidad el desarrollo web está de moda por la facilidad que brinda de ser usado en cualquier dispositivo y en cualquier lugar, solo con un navegador y una conexión a internet. La mayoría de las aplicaciones más usadas a nivel mundial (\href{instagram.com}{Instagram}, \href{youtube.com}{Youtube}, \href{https://web.whatsapp.com/}{Whatsapp}) cuentan con una versión web.
\newline

Una aplicación web consta de 2 partes, el \textbf{Backend} y el \textbf{Frontend}. El \textbf{frontend} es la parte que corre en el dispositivo del usuario final, y se encarga de mostrar una interfaz con la que este puede interactuar para conseguir un objetivo. El \textbf{backend}, por otro lado, es la parte que corre en servidores y se encarga de la lógica del negocio, la seguridad, la persistencia de los datos, etc. El \textbf{frontend} interactúa con el \textbf{backend} para guardar, modificar, o acceder a información.
\newline

Los días en que bastaba \textit{HTML}, \textit{CSS} y \textit{Javascript Vanilla} (Javascript sin bibliotecas ni frameworks de terceros) para construir el frontend de una aplicación quedaron atrás. La complejidad de los productos actuales ha hecho que sea prácticamente imposible mantener software con estas tecnologías, por lo que los frameworks y bibliotecas de Javascript son prácticamente obligatorios a la hora de desarrollar una buena aplicación.
\newline

Según una encuesta desarrollada por StackOverflow \cite{encuesta2022} en 2022 los frameworks y/o bibliotecas más usados en el desarrollo frontend son: \textit{React.js}, \textit{jQuery}, \textit{Angular} y \textit{Vue.js}.
\newline

Para estilizar las páginas se utiliza el lenguaje \textit{CSS}, pero debido a la naturaleza del mismo, en aplicaciones grandes el código \textit{CSS} crece demasiado y se hace muy difícil de mantener, por eso en algunas ocasiones los desarrolladores optan por usar frameworks que eviten de cierta manera estas complicaciones, dos de los más usados en la actualidad son \href{getbootstrap.com}{Bootstrap} [cita], y \href{tailwindcss.com}{Tailwind CSS} \cite{tailwind}.
\newline

En el caso del backend existe una mayor variedad de tecnologías, por lo que los desarrolladores suelen elegir la que más se adecúa al proyecto que buscan desarrollar, en estos casos es casi imprescindible utilizar un framework, ya que las bases de código en aplicaciones no tan complejas suele crecer muchísimo por lo que puede haber problemas en el mantenimiento de las mismas y pueden surgir agujeros de seguridad.
\newline

Según la encuesta de StackOverflow del 2022 \cite{encuesta2022} estos son los lenguajes de programación más usados: 
\begin{enumerate}
	\item JavaScript
	\item Python
	\item TypeScript
	\item Java
	\item Bash/Shell
	\item C\#
	\item C++
	\item PHP
	\item PowerShell
	\item Go
	\item Rust
\end{enumerate}
No se incluyeron en la lista lenguajes de marcado y estilos (\textit{HTML} y \textit{CSS}) ni lenguajes de consulta (\textit{SQL})
\newline

Para almacenar los datos en una aplicación se utilizan motores de base de datos, según la encuesta de StackOverflow de 2022 \cite{encuesta2022} estos son los más utilizados:
\begin{enumerate}
	\item MySQL
	\item PostgreSQL
	\item SQLite
	\item MongoDB
	\item Microsoft SQL Server
	\item Redis
\end{enumerate}

Al igual que pasa con \textit{CSS}, el código de base de datos en aplicaciones grandes suele crecer muchísimo, hasta el punto que se convierte prácticamente en otro proyecto para mantener, para resolver este problema existen los ORM (Object–relational mapping) \cite{orm}, una técnica de programación para convertir datos entre sistemas de tipos usando lenguajes de programación orientado a objetos. Esto crea una base de datos virtual que puede ser usada desde un lenguaje de programación. Hay una gran cantidad de ORM's en todos (o casi todos) los lenguajes de programación.
\newline

Otro de los problemas habituales a la hora de desarrollar una aplicación son las diferencias entre los entornos de desarrollo y producción, en ocasiones las diferencias de sistemas operativos y/o versiones de paquetes instalados entre uno y el otro pueden causar inconvenientes. Afortunadamente hay una herramienta que hace que el desarrollo de una aplicación sea más eficiente y predecible, eliminando tareas de configuración repetitivas, estoy hablando de \href{docker.com}{Docker} \cite{docker_docs}. Según la encuesta de 2022 de Stackoverflow \cite{encuesta2022} Docker es la herramienta más amada por los desarrolladores.

\section{Sistemas de resolución de dudas relevantes}

Existen sistemas de resolución de preguntas que traen ideas novedosas que pueden ser tenidas en cuenta, a continuación se presentan algunos de ellos:

\subsection{NSIR}

\textbf{NSIR} es un sistema desarrollado por la Universidad de Michigan que responde preguntas de forma automática. Dada una pregunta \textbf{NSIR} obtiene los principales resultados devueltos por motores de búsqueda (\href{google.com}{Google}, \href{yahoo.com}{Yahoo}, etc), los analiza y extrae de estos un conjunto de posibles respuestas, luego estas son evaluadas usando novedosas métricas, y de ahí retornan al usuario las respuestas con mejor calificación \cite{nsir}.


\subsection{Schema2QA}

\textbf{Schema2QA} es una herramienta de código abierto para generar sistemas de preguntas y respuestas a partir de un esquema de base de datos con unas pocas anotaciones sobre sus campos \cite{s2qa}. Este sistema intenta cubrir el espacio de consultas con un gran número de preguntas del dominio. Luego los datos obtenidos son usados para entrenar un modelo del lenguaje basado en \textit{BERT} \cite{bert}. Este sistema ha sido utilizado en dominios como los restaurantes, libros, música, entre otros, y alcanzado muy buenos resultados \cite{s2qa}.


\subsection{SmartQ}

\textbf{SmartQ} es un sistema de preguntas y respuestas basado en la reputación de los usuarios. Propone un mecanismo inteligente en la que hay usuarios especializados en diversas temáticas, y cuando alguien hace una pregunta esta es clasificada y enviada a personas expertas en el tema. Los usuarios van construyendo una reputación a medida que aportan respuestas valiosas a lo largo del tiempo. En el sistema también existe un mecanismo de seguidores en el que algunos usuarios expresan su confianza en otros, además de un sistema anti spam \cite{smartq}.


\chapter{Propuesta}\label{chapter:proposal}

El proyecto está compuesto por una REST API desarrollada en \href{https://gin-gonic.com/}{Gin} y un frontend desarrollado con \href{https://nuxtjs.org/}{Nuxt.js}.
\newline

Para el backend se usará una base de datos \href{https://www.postgresql.org/}{PostgreSQL} que será manejada a través del \textit{ORM} \href{https://gorm.io/}{Gorm}, se usará un sistema de \textit{JWT} para la autenticación, y los tokens de los usuarios se almacenarán en memoria usando una base de datos \href{https://redis.io/}{Redis}.
\newline

Para el frontend estaremos usando la \href{https://nuxtjs.org/}{Nuxt.js} en su versión 3.0, y nos apoyaremos de \href{https://tailwindcss.com/}{Tailwind CSS} para estilizar las vistas.
\chapter{Detalles de Implementación}\label{chapter:implementation}

En el presente capítulo se presentan detalles de la implementación de la solución propuesta anteriormente, pasando por las tecnologías empleadas, la estructura de la base de datos, el backend y terminando con el frontend.
\newline

\section{Tecnologías empleadas}

(figura \ref{fig:arquitecture}) Para conseguir desacoplar al máximo el desarrollo, el proyecto se compuso de una REST API desarrollada en \href{https://gin-gonic.com/}{Gin} y un frontend desarrollado con \href{https://nuxtjs.org/}{Nuxt.js}.
\newline

Para tener el mecanismo de usuarios con roles, áreas, así como almacenar las preguntas, clasificaciones, y respuestas, hace falta una base de datos, en este caso se usó \href{https://www.postgresql.org/}{PostgreSQL} que fue manejada a través del \textit{ORM} \href{https://gorm.io/}{Gorm} para facilitar el desarrollo. No todos los usuarios pueden acceder a todas las funcionalidades, por eso se empleó un mecanismo de autenticación y autorización con tokens \textit{JWT}, y para manejar la caché en el servidor se usó la base de datos en memoria \textit{Redis}.
\newline


\begin{figure}[h]
	\includegraphics[width=15cm, height=11.25cm]{arquitecture.png}
	\caption{Arquitectura de la aplicación}
	\label{fig:arquitecture}
\end{figure}


\section{Estructura de la Base de Datos}
La base de datos es uno de los elementos fundamentales en la implementación, pues la forma en la que se almacenan los datos influye mucho en lo que se quiere lograr con la aplicación. Como se mencionó en capítulos anteriores cada usuario debe tener un rol; cada uno de los 5 roles: \textbf{estudidante}, \textbf{clasificador}, \textbf{especialista nivel 1}, \textbf{especialista nivel 2} y \textbf{administrador} se guardaron en una tabla \textbf{roles} (tabla \ref{table:roles}). Cada especialista debe formar parte de un \textbf{área} (\ref{table:areas}), por lo que se creó una tabla para guardar todas las áreas de la universidad. La tabla \textbf{usuario} (\ref{table:users})  se encarga de guardar los datos de las personas registradas en la plataforma, tales como correo y contraseña (encriptada), también tiene una referencia al \textbf{rol} y el \textbf{área} a la que pertenece dicho usuario, en caso de que no sea un especialista la referencia del área es nula. La tabla \textbf{preguntas} (\ref{table:questions}) contiene el texto de la pregunta, una referencia al usuario autor de la misma, además de una llave foránea del área en la cual fue clasificada, otra al usuario responsable de darle respuesta y el texto de la respuesta, estos últimos 3 campos pueden ser nulos y cambiar con el tiempo en dependencia del estado de la pregunta. Los mensajes de chat se guardan en una tabla (\ref{table:chat}) con una referencia a \textbf{preguntas}, una al autor del mensaje, además cuenta con un campo para guardar el texto del mismo, y un booleano que indica si fue o no leído.

\begin{table}[h]
	\begin{tabular}{| c | c | c | c | c |}
		\hline
		Campo & Tipo & Llave Primaria & Llave Foránea (Referencia a) \\ \hline
		Id & Entero & Sí & -  \\ \hline 
		Nombre & String & No & - \\ \hline
	\end{tabular}
	\caption{Tabla roles}
	\label{table:roles}
\end{table}


\begin{table}[h]
	\begin{tabular}{| c | c | c | c | c |}
		\hline
		Campo & Tipo & Llave Primaria & Llave Foránea (Referencia a) \\ \hline
		Id & Entero & Sí & -  \\ \hline 
		Nombre & String & No & - \\ \hline
	\end{tabular}
	\caption{Tabla áreas}
	\label{table:areas}
\end{table}

\begin{table}[h]
	\begin{tabular}{| c | c | c | c | c |}
		\hline
		Campo & Tipo & Llave Primaria & Llave Foránea (Referencia a) \\ \hline
		Id & Entero & Sí & -  \\ \hline 
		Nombre & String & No & - \\ \hline
		Correo & String & No & - \\ \hline
		Contraseña & String & No & - \\ \hline
		Id Role & Entero & No & Sí (Rol) \\ \hline
		Id Área & Entero & No & Sí (Área) \\ \hline
	\end{tabular}
	\caption{Tabla usuarios}
	\label{table:users}
\end{table}

\begin{table}[h]
	\begin{tabular}{| c | c | c | c | c |}
		\hline
		Campo & Tipo & Llave Primaria & Llave Foránea (Referencia a) \\ \hline
		Id & Entero & Sí & -  \\ \hline 
		Texto & String & No & - \\ \hline
		Respuesta & String & No & - \\ \hline
		Autor & Entero & No & Sí (Usuarios) \\ \hline
		Responsable & Entero & No & Sí (Usuarios) \\ \hline
	\end{tabular}
	\caption{Tabla preguntas}
	\label{table:questions}
\end{table}

\begin{table}[h]
	\begin{tabular}{| c | c | c | c | c |}
		\hline
		Campo & Tipo & Llave Primaria & Llave Foránea (Referencia a) \\ \hline
		Id & Entero & Sí & -  \\ \hline 
		Texto & String & No & - \\ \hline
		Leído & Booleano & No & - \\ \hline
		Id Pregunta & Entero & No & Sí (Preguntas) \\ \hline
		Autor & Entero & No & Sí (Usuarios) \\ \hline
	\end{tabular}
	\caption{Tabla mensajes de chat}
	\label{table:chat}
\end{table}


\section{Backend}

\subsection{Estructura:}
El backend se estructura por capas:
\newline

\includegraphics[width=13.8cm, height=2.5cm]{structure_backend.png}

La capa \textbf{Handler} es la encargada de interceptar los requests, interactuar con los permisos, detectar posibles errores dentro del request recibido, preparar las estructuras necesarias para finalmente ejecutar alguno(s) de los servicios de la capa \textbf{Service} para luego retornar el response adecuado para el request.
\newline

La capa \textbf{Service} consta de varios servicios tales como iniciar sesión, registrarse, etc, esta se encarga de ejecutar los pasos necesarios para la acción requerida, abstrayéndose de las validaciones del requests (porque ya fue hecho por la capa \textbf{Handler}), para obtener datos deberá pedírselos a la capa \textbf{Repository} y luego retornarlos a la capa \textbf{Handler}.
\newline

La capa \textbf{Repository} es la encargada de las operaciones con la base de datos, para ello recibe indicaciones de la capa \textbf{Serivice} y auxiliándose del \textit{ORM} \href{gorm.io}{Gorm} y inserta, modifica, lee, y/o elimina datos y le entrega una respuesta a la capa \textbf{Service}.
\newline

La capa \textbf{Models} tiene, con la sintaxis de estructuras de \href{go.dev}{Go}, se describen todas las entidades y las relaciones de la base de datos, para ello hace uso del \textit{ORM} \href{gorm.io}{Gorm}.

\subsection{Usuarios}

\subsubsection{Crear una cuenta}

Para crear un nuevo usuario se tiene el endpoint de tipo \textbf{POST} a la la url \textit{\textcolor{red}{/api/account/signup}}, el cuerpo de la petición debe tener la siguiente forma:

\begin{lstlisting}[language=javascript]
	{
		"email",
		"name",
		"pass",
		"worker"
	}
\end{lstlisting}

El campo \textit{worker} especifica el rol con el que el usuario va a ser creado inicialmente (que luego puede ser cambiado por el administrador), si se indica que \textit{worker = 0} entonces el rol del usuario será \textbf{estudiante}, de lo contrario será \textbf{clasificador}.
\newline

Luego de validar todos los campos se llama a la función \textit{Signup} del \textit{user\_service}, esta se encarga de encriptar la contraseña para luego llamar a la función \textit{Create} del \textit{user\_repository} cuya tarea es insertar estos datos en la tabla \textit{user}. Para encriptar la contraseña se usa el algoritmo \textit{SHA256} con un salt aleatorio de 32 bits. El salt es guardado junto con la contraseña para poder usarlo a la hora de desencriptarla.
\newline

Luego de esto se genera un token con los datos del usuario usando \textit{JWT} que es devuelto en el response. Este token deberá proveerse en muchos request que requieran autenticación.
\newline

\subsubsection{Iniciar sesión}
Para iniciar sesión se tiene el endpoint de tipo \textbf{POST} con url \textit{\textcolor{red}{/api/account/signin}}, el cuerpo de la petición debe tener la siguiente forma:

\begin{lstlisting}[language=javascript]
	{
		"email",
		"pass"
	}
\end{lstlisting}

En esta ocasión el handler va a llamar a la función \textit{Signin} de \textit{user\_service}, la cual se va a encargar de encontrar a ese usuario en la base de datos usando la función \textit{FindByEmail} de \textit{user\_repository} y validar la contraseña, si todo está en orden se va a generar un token para retornar en el response.
\newline

\textbf{Crearse una cuenta} e \textbf{iniciar sesión} son las dos únicas operaciones que se pueden hacer sin estar autenticado, en el resto de endpoints se requiere de autenticación, para ello se hace uso del middleware \textit{auth\_user} que verifica si en el header \textit{Authentication} del request hay un token válido, este header debe tener el formato \textit{\textcolor{blue}{Bearer 'token'}}
\newline

\subsubsection{Middlewares}

A parte del middleware \textit{auth\_user} mencionado anteriormente también se desarrollaron otros 2 que manejan la autorización de un usuario a un recurso dependiendo de su rol:

\begin{lstlisting}[language=go]
	OnlyRoles(roles []string)
\end{lstlisting}

Este middleware recibe un array de strings con los nombres de los roles que deben ser autorizados a acceder al recurso que se quiere proteger.

\begin{lstlisting}[language=go]
	NotTheseRoles(roles []string)
\end{lstlisting}

Por el contrario, este último recibe los nombres de los roles que no deben ser autorizados a determinado recurso que se quiere proteger.
\newline

Se pueden usar indistintamente para la misma función, dependiendo del caso, el primero es más cómodo cuando hay pocos roles a autorizar, y el último casi siempre se usa cuando hay un determinado rol que se quiere desautorizar. 
\newline

\subsubsection{Modificar rol}

Para modificar el rol de un determinado usuario se creó el endpoint de tipo \textbf{PUT} \textit{\textcolor{red}{/api/users/update-role}}, cuyo cuerpo debe tener la siguiente estructura:

\begin{lstlisting}[language=javascript]
		{
			"user_id"
			"role_id"
		}
\end{lstlisting}

Este endpoint solo puede ser accedido por usuarios con el rol de administrador, y le modifica al usuario con \textit{id = 'user\_id'} su rol al rol con \textit{id = 'role\_id'}

\subsubsection{Modificar área}

Para esto se creó el endpoint de tipo \textbf{PUT} \textit{\textcolor{red}{/api/users/update-area}}, en el cuerpo se debe especificar el id del usuario y el id del área que se le quiere asignar, las áreas se debe crear previamente.
\newline

Para crear un área se usa el endpoint de tipo \textbf{POST} \textit{\textcolor{red}{/api/areas/add}} especificando el nombre en el cuerpo. Estas acciones solo pueden ser llevadas a cabo por administradores.

\subsection{Preguntas}

Las preguntas se pueden crear, clasificar, tomar, responder y subir de nivel. A continuación se describen los endpoints para realizar este tipo de acciones.
\newline

Las preguntas solo pueden ser creadas por estudiantes usando el endpoint de tipo \textbf{POST} \textit{\textcolor{red}{/api/questions/add}}, se debe añadir el texto de la misma en el cuerpo de la petición. Para clasificarlas se usa el endpoint \textbf{PUT} \textit{\textcolor{red}{/api/questions/clasify}}, aclarando en su cuerpo el id de la pregunta y el id del área a la que se quiere clasificar, esta acción solo puede ser llevada a cabo por clasificadores. Una vez la pregunta es clasificada en un área \textit{X}, está lista para que los especialistas de nivel 1 de \textit{X} puedan responderla, para que no haya colisiones entre varios usuarios intentando responder la misma pregunta, se creó un endpoint para \textit{tomar} la pregunta, esta acción quita a la pregunta de la lista de pendientes y la asigna al usuario que la haya tomado, el endpoint responsable de ejecutar esta tarea es \textit{\textcolor{red}{/api/questions/take}} (\textbf{PUT}), en cuyo cuerpo se debe agregar la pregunta a tomar. Si un especialista de nivel 1 tomó una pregunta que no puede responder, entonces debe subirla de nivel con el endpoint de tipo \textbf{PUT} \textit{\textcolor{red}{/api/questions/up-level}}, y en ese entonces estará lista para ser tomada por un especialista de nivel 2, y una vez tomada puede ser elevada a la administración usando el mismo endpoint. Para responder una pregunta previamente se debe tomar, y luego usar el endpoint de tipo \textbf{PUT} \textit{\textcolor{red}{/api/questions/response}}, especificando en el cuerpo el id de la pregunta y el texto de la respuesta.

\subsection{Chat}

Por cada pregunta se abre un chat en la que el estudiante y el responsable de su pregunta pueden intercambiar mensajes, solo pueden escribir y ver el chat el autor de la pregunta y el especialista o administrador que la haya tomado.
\newline

Para ver los mensajes se usa el endpoint \textbf{GET} \textit{\textcolor{red}{/api/chat/'id\_pregunta'}}, y para mandar un mensaje \textit{\textcolor{red}{/api/chat'}} (\textbf{POST}) con el id de la pregunta y el texto del mensaje en el cuerpo de la petición.

\section{Frontend}

La aplicación del lado del cliente puede estar en dos estados, \textbf{sin iniciar sesión} o \textbf{sesión iniciada}. En caso de que no haya una sesión iniciada se mostrará una página de bienvenida y el usuario tendrá acceso a una página de creación de cuenta (figura \ref{fig:signup}) y otra de inicio de sesión (figura \ref{fig:login}). De lo contrario dependerá del rol del usuario las vistas y acciones que tendrá disponible: en caso de que sea \textbf{estudiante} (tabla \ref{table:student_views}) podrá ver su historial de preguntas, escribir una nueva, abrir un chat por cada una de ellas y escribir mensajes. El \textbf{clasificador} (tabla \ref{table:clasifier_views}) solo puede clasificar las dudas en alguna de las áreas existentes. Los \textbf{especialistas} (tabla \ref{table:specialist_views}) observarán las dudas clasificadas en su área, y de ahí podrán tomarlas para luego responderlas, subirlas de nivel o abrir un chat mediante el cual conversar con el autor de la duda. Los \textbf{administradores} (tabla \ref{table:admin_views}) tendrán acceso al listado de usuarios mediante el cual podrán cambiar de rol y de área a cada uno de ellos, también en el listado de áreas se les habilitará la opción de crear nuevas, y en el listado de preguntas subidas a la administración podrán tomarlas para luego responderlas o chatear con los autores.
\newline

\begin{figure}[h]
\begin{center}
	\includegraphics[width=12cm, height=12cm]{signup_page.png}
	\caption{Página de registro}
	\label{fig:signup}
\end{center}
	
\end{figure}

\begin{figure}[h]
	\begin{center}
		\includegraphics[width=12cm, height=12cm]{login_page.png}
		\caption{Página de inicio de sesión}
		\label{fig:login}
	\end{center}
	
\end{figure}

	\begin{table}[h]
		\begin{center}
		\begin{tabular}{| c | c |}
			\hline
			Vistas & Acciones \\ \hline
			Formulario de duda & Escribir duda  \\ \hline 
			Su historial de dudas & Ver respuestas, abrir chat  \\ \hline
			Chat & leer y escribir mensajes \\ \hline
			
		\end{tabular}
		\caption{Tabla de vistas y acciones permitidas a estudiantes}
		\label{table:student_views}
		
	\end{center}
	\end{table}


	\begin{table}[h]
	\begin{center}
		\begin{tabular}{| c | c |}
			\hline
			Vistas & Acciones \\ \hline
			Preguntas sin clasificar & clasificar  \\ \hline 
		\end{tabular}
		\caption{Tabla de vistas y acciones permitidas a clasificadores}
		\label{table:clasifier_views}
		
	\end{center}
\end{table}


	\begin{table}[h]
	\begin{center}
		\begin{tabular}{| c | c |}
			\hline
			Vistas & Acciones \\ \hline
			Preguntas clasificadas en su área & Tomar pregunta  \\ \hline 
			Preguntas Tomadas & Responder, abrir chat, subir de nivel  \\ \hline 
			Chat & Leer y escribir mensajes  \\ \hline 
		\end{tabular}
		\caption{Tabla de vistas y acciones permitidas a especialistas}
		\label{table:specialist_views}
		
	\end{center}
\end{table}

	\begin{table}[h]
	\begin{center}
		\begin{tabular}{| c | c |}
			\hline
			Vistas & Acciones \\ \hline
			Listado de áreas & Crear nueva área \\ \hline 
			Listado de usuarios & Cambiarlos de rol o área  \\ \hline 
			Preguntas subidas a la administración & Tomar pregunta \\ \hline 
			Preguntas tomadas & Responder y abrir chat  \\ \hline 
			Chat & Leer y escribir mensajes  \\ \hline 
		\end{tabular}
		\caption{Tabla de vistas y acciones permitidas a administradores}
		\label{table:admin_views}
		
	\end{center}
\end{table}

Para la autenticación almacenamos los datos del usuario y su token \textit{JWT} en las cookies de la página para poder acceder a estos y usarlos en los requests que así lo requieran, también es importante conocer el rol y área para saber a qué páginas y recursos dicho el usuario debe tener acceso.
\newline

Toda la lógica de los llamados a la API del backend se separó en la carpeta \textit{'api'} para mantenerla desacoplada de los componentes visuales.


\chapter{Pruebas}\label{chapter:testing}

A continuación se describen las pruebas realizadas para validar el sistema implementado, las pruebas se dividieron en 2 tipos: backend y frontend. Para las del backend se crearon tests automáticos probando varios de los endpoints de la aplicación, en el caso del frontend se hicieron varios experimentos interactuando directamente con la interfaz web.

\section{Backend}

Para comprobar que todo funcionaba correctamente se prepararon varias pruebas, se crearon 500 usuarios: 100 con nombres \textit{'estudiante1'}, \textit{'estudiante2}, ... , \textit{estudiante100}, otros 100 con nombre \textit{'clasificador1'}, ... \textit{clasificador100}, y así con el resto de los roles, luego de esto se verificó que existían 501 usuarios, un administrador que se crea automáticamente al ejecutar el backend por primera vez, y los 500 creados en la prueba, se verificó que existían 100 de cada grupo leyendo los prefijos en sus nombres.
\newline

Luego, usando el administrador creado por defecto, se le cambió el rol a cada usuario dependiendo de su nombre, a los que tenían prefijo 'estudiante' se le asignó dicho rol, igualmente con el resto de los roles. Al concluir se verificó que los prefijos en los nombres coincidían con el rol de cada usuario.
\newline

También se crearon satisfactoriamente 5 áreas, llamadas \textit{'area1'}, \textit{'area2'}, ... y \textit{'area5'}, y se le asignó a cada especialista con \textit{'id mod 5 = i'} el \textit{'areai'}, la operación fue realizada con éxito.
\newline

Luego, con cada estudiante se escribió una pregunta, y con un clasificador se clasificó cada pregunta siguiendo el mismo patrón de las áreas de los especialistas. Con un especialista de cada área se tomaron las preguntas satisfactoriamente, las que provenían de estudiantes con id par se respondieron, y las de otras se subieron de nivel; con estas últimas se aplicó un procedimiento similar respondiendo las de estudiantes con id divisible por 4 y subiendo a la administración el resto para finalmente darles respuestas. Se comprobó durante todo el proceso que las tareas iban según lo planeado.
\newline

\section{Frontend}

En este caso se prepararon varias tareas (se muestran algunas capturas de pantalla):
\begin{itemize}
	\item Registrarse con un usuario llamado f\_estudiante: se debe entrar a la página, presionar en el botón \textit{'Registrarse'} y rellenar los datos (al desmarcar la opción de \textit{'Soy trabajador de MATCOM'} el usuario es creado en el rol de estudiante) (figura \ref{fig:f_estudiante_registro}).
	
	\item Registrarse con un usuario llamado f\_clasificador: se debe cerrar sesión y aplicar el mismo procedimiento del punto anterior (al marcar la opción \textit{'Soy trabajador de MATCOM'} el usuario es creado con el rol de clasificador).
	
	\item Registrarse con otros 2 usuarios, uno llamado f\_especialista1 y otro f\_especialista2.
	
	\item Iniciar sesión con el administrador y asignarle el rol \textit{'Especialista Nivel 1'} al usuario con nombre f\_especialista1, y \textit{'Especialista Nivel 2'} a f\_especialista2: para esto se debe cerrar sesión, iniciar sesión con el administrador (correo: admin@admin.com, contraseña: administrador), ir a la sección de usuarios, buscar al usuario en cuestión y cambiarle el rol (figura \ref{fig:f_cambio_rol}).
	
	\item  Crear un área con el nombre \textit{'f\_area'}: esto se hace en la sección de áreas (figura \ref{fig:f_crear_ara}).
	
	\item Asignarle el área \textit{f\_area} a los usuarios f\_especialista1 y f\_especialista2, esto se lleva a cabo en la sección de usuarios (figura \ref{fig:f_cambiar_area}).
	
	\item Iniciar sesión con f\_estudiante y escribir 3 preguntas (figura \ref{fig:f_crear_pregunta}). Luego entrar a la sección \textit{'Mis dudas'} y verificar que estén creadas las preguntas (figura \ref{fig:f_verificar_crear_preguntas}).
	
	\item Iniciar sesión con f\_clasificador y clasificar las 3 preguntas creadas en el punto anterior en el área \textit{'f\_area'} (figura \ref{fig:f_clasificar_preguntas}).
	
	\item Iniciar sesión con f\_especialista1 y tomar las 3 preguntas, esto se hace desde la sección \textit{'Preguntas clasificadas'} (figura \ref{fig:f_tomar_preguntas1}). Luego se debe ir a la sección \textit{'Mis preguntas'} y verificar que las dudas estén ahí (figura \ref{fig:f_virificar_preguntas_tomadas1}).
	
	\item Responder la pregunta \textit{'pregunta 1'} y subir de nivel las otras 2 (figura \ref{fig:f_subir_nivel_2}). Luego iniciar sesión con f\_estudiante y verificar que tiene respuesta en la pregunta 1 (figura \ref{fig:f_verificar_respuesta1}), también se debe ingresar con f\_especialista2 y ver si le aparecen las preguntas \textit{'pregunta 2'} y \textit{'pregunta 3'} (figura \ref{fig:f_verificar_subida2}).
	
	\item Iniciar sesión con f\_especialista2 y tomar las 2 preguntas (pregunta 1 y pregunta 2), responder la 2 y subir de nivel la 3 (figura \ref{fig:f_subida3}). Verificar con f\_estudiante que la pregunta 2 tiene respuesta (figura \ref{fig:f_respuesta2}) y con el administrador que le aparece la pregunta 3 (figura \ref{fig:f_verificacion_admin}).
	
	\item Con el administrador tomar la pregunta 3, chatear con el estudiante (figura \ref{fig:f_chat_admin}) y finalmente darle respuesta (figura \ref{fig:f_respuesta3}).
\newline

	
Todos los tests pasaron satisfactoriamente por lo que el sistema está listo para ser lanzado.
	
\end{itemize}


	\begin{figure}[h]
		\begin{center}
		\includegraphics[width=6cm, height=5cm]{registro_f_estudiante.png}
		\caption{Registro de f\_estudiante}
		\label{fig:f_estudiante_registro}
		
	\end{center}
	\end{figure}

	\begin{figure}[h]
	\begin{center}
		\includegraphics[width=6cm, height=5cm]{cambio_rol.png}
		\caption{Cambio de rol}
		\label{fig:f_cambio_rol}
		
	\end{center}
\end{figure}

	\begin{figure}[h]
	\begin{center}
		\includegraphics[width=6cm, height=5cm]{crear_area.png}
		\caption{Crear área}
		\label{fig:f_crear_ara}
		
	\end{center}
\end{figure}

	\begin{figure}[h]
	\begin{center}
		\includegraphics[width=6cm, height=5cm]{cambiar_area.png}
		\caption{Cambiar área}
		\label{fig:f_cambiar_area}
		
	\end{center}
\end{figure}

\begin{figure}[h]
	\begin{center}
		\includegraphics[width=6cm, height=5cm]{crear_pregunta.png}
		\caption{Crear preguntas}
		\label{fig:f_crear_pregunta}
		
	\end{center}
\end{figure}

\begin{figure}[h]
	\begin{center}
		\includegraphics[width=6cm, height=5cm]{verificar_crear_preguntas.png}
		\caption{Verificación de la creación de las preguntas}
		\label{fig:f_verificar_crear_preguntas}
		
	\end{center}
\end{figure}

\begin{figure}[h]
	\begin{center}
		\includegraphics[width=6cm, height=5cm]{clasificar_preguntas.png}
		\caption{Clasificar preguntas}
		\label{fig:f_clasificar_preguntas}
		
	\end{center}
\end{figure}

\begin{figure}[h]
	\begin{center}
		\includegraphics[width=6cm, height=5cm]{tomar_preguntas.png}
		\caption{Tomar preguntas con el especialista de nivel 1}
		\label{fig:f_tomar_preguntas1}
		
	\end{center}
\end{figure}

\begin{figure}[h]
	\begin{center}
		\includegraphics[width=6cm, height=5cm]{verificar_preguntas_tomadas.png}
		\caption{Verificación de que las preguntas fueron tomadas}
		\label{fig:f_virificar_preguntas_tomadas1}
		
	\end{center}
\end{figure}

\begin{figure}[h]
	\begin{center}
		\includegraphics[width=6cm, height=5cm]{subir_nivel_2.png}
		\caption{Subida de nivel de las preguntas 2 y 3}
		\label{fig:f_subir_nivel_2}
		
	\end{center}
\end{figure}

\begin{figure}[h]
	\begin{center}
		\includegraphics[width=6cm, height=5cm]{verificar_respuesta_1.png}
		\caption{Verificación de la respuesta a la pregunta 1}
		\label{fig:f_verificar_respuesta1}
		
	\end{center}
\end{figure}

\begin{figure}[h]
	\begin{center}
		\includegraphics[width=6cm, height=5cm]{verificar_subida_2.png}
		\caption{Verificación de la subida de nivel de las preguntas 2 y 3}
		\label{fig:f_verificar_subida2}
		
	\end{center}
\end{figure}

\begin{figure}[h]
	\begin{center}
		\includegraphics[width=6cm, height=5cm]{subida_nivel3.png}
		\caption{Subida a la administración de las preguntas 2 y 3}
		\label{fig:f_subida3}
		
	\end{center}
\end{figure}

\begin{figure}[h]
	\begin{center}
		\includegraphics[width=6cm, height=5cm]{respuesta2.png}
		\caption{Verificación de la respuesta a la pregunta 2}
		\label{fig:f_respuesta2}
		
	\end{center}
\end{figure}

\begin{figure}[h]
	\begin{center}
		\includegraphics[width=6cm, height=5cm]{verificar_subida3.png}
		\caption{Verificación de la subida a la administración de la pregunta 3}
		\label{fig:f_verificacion_admin}
		
	\end{center}
\end{figure}


\begin{figure}[h]
	\begin{center}
		\includegraphics[width=6cm, height=5cm]{chat_admin.png}
		\caption{Chat entre el administrador y f\_estudiante}
		\label{fig:f_chat_admin}
		
	\end{center}
\end{figure}

\begin{figure}[h]
	\begin{center}
		\includegraphics[width=6cm, height=5cm]{respuesta3.png}
		\caption{Verificación de la respuesta a la pregunta 3}
		\label{fig:f_respuesta3}
		
	\end{center}
\end{figure}


\backmatter

\begin{conclusions}
    De los principales softwares de reportes e incidencias existentes en el mercado, ninguno se adecuaba a los propósitos de la Universidad, por lo que se desarrolló desde cero una plataforma que puede ser editada en un futuro si se estima pertinente, se utilizaron algunas de las tecnologías punteras en la actualidad, es de fácil uso, tanto para el personal calificado como para los estudiantes. Al ser una solución web puede es accesible desde cualquier lugar con un dispositivo conectado a internet. Con ella se soluciona el problema de organización existente en los grupos de Telegram y Whatsapp, las dudas llegan a las personas capacitadas para darles respuesta y los especialistas más experimentados ahorran tiempo que pueden dedicar a las dudas más complicadas al no recibir incidencias comunes que pueden ser solucionadas por personal menos experimentado. 
\end{conclusions}

\begin{recomendations}
    El software desarrollado fue sometido a varias pruebas y pasó todas con éxito, aún así toda pieza de software está expuesta a posibles errores por lo que al lanzarse la plataforma se recomienda monitorear su funcionamiento ante posibles fallos que puedan ocurrir.
    \newline
    
    Es importante mantenerse informado en los avances que puedan existir en materia de desarrollo de software y de sistemas parecidos en el mundo, para poder adaptar la plataforma con las innovaciones que vayan saliendo a la luz y tener el mejor producto posible para los estudiantes de la universidad.
    \newline
    
    Acumular los datos para en un futuro tener material y poder entrenar modelos de inteligencia artificial para automatizar tareas como la clasificación de dudas y la respuestas a preguntas repetitivas. Es recomendable también mirar las tendencias de machine learning porque es un campo que está creciendo muy rápido y van a surgir tecnologías cada vez más potentes que pueden ser utilizadas apara mejorar el rendimiento del sistema.
\end{recomendations}

\printbibliography[heading=bibintoc]
\begin{thebibliography}{X}
	\bibitem{apirest} Documentación de Integración de Aplicaciones de Amazon
	
	
	\url{https://aws.amazon.com/es/what-is/api/#:~:text=API%20significa%20%E2%80%9Cinterfaz%20de%20programaci%C3%B3n,de%20servicio%20entre%20dos%20aplicaciones.}


	\bibitem{frontend_vs_backend} Qué es Frontend y Backend: diferencias y características - Platzi
	
	\url{https://platzi.com/blog/que-es-frontend-y-backend/}
	
	\bibitem{encuesta2022} Encuesta de StackOverflow 2022
	
	
	\url{https://survey.stackoverflow.co/2022/#most-popular-technologies-webframe}
	
\bibitem{vue} Documentación oficial de Vue

\url{https://vuejs.org/guide/introduction.html}

\bibitem{nuxt} Documentación oficial de Nuxt 3

\url{https://v3.nuxtjs.org/guide/concepts/auto-imports}

\bibitem{tailwind} Página oficial de Tailwind CSS

\url{https://tailwindcss.com/}

\bibitem{golang} Documentación oficial de Go

\url{https://go.dev/doc/}

\bibitem{gin} Documentación oficial de Gin

\url{https://gin-gonic.com/docs/}

\bibitem{postgres} Página oficial de PostgreSQL

\url{https://www.postgresql.org/}

\bibitem{redis} Página oficial de Redis

\url{https://redis.io/}

\bibitem{jwt} \url{https://jwt.io/}

\bibitem{docker} Página oficial de Docker

\url{https://www.docker.com/}

\bibitem{docker_docs} Documentación oficial de Docker

\url{https://docs.docker.com/get-started/}

\bibitem{orm} What is Object/Relational Mapping?. Hibernate Overview. JBOSS Hibernate. Retrieved 27 January 2022.

\bibitem{gorm} Documentación oficial de Gorm

\url{https://gorm.io/docs/}
	
\end{thebibliography}

\end{document}